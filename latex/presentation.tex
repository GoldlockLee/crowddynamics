%%%%%%%%%%%%%%%%%%%%%%%%%%%%%%%%%%%%%%%%%
% Beamer Presentation
% LaTeX Template
% Version 1.0 (10/11/12)
%
% This template has been downloaded from:
% http://www.LaTeXTemplates.com
%
% License:
% CC BY-NC-SA 3.0 (http://creativecommons.org/licenses/by-nc-sa/3.0/)
%
%%%%%%%%%%%%%%%%%%%%%%%%%%%%%%%%%%%%%%%%%

%----------------------------------------------------------------------------------------
%	PACKAGES AND THEMES
%----------------------------------------------------------------------------------------

\documentclass{beamer}

\mode<presentation> {

% The Beamer class comes with a number of default slide themes
% which change the colors and layouts of slides. Below this is a list
% of all the themes, uncomment each in turn to see what they look like.

%\usetheme{default}
%\usetheme{AnnArbor}
%\usetheme{Antibes}
%\usetheme{Bergen}
%\usetheme{Berkeley}
%\usetheme{Berlin}
%\usetheme{Boadilla}
%\usetheme{CambridgeUS}
%\usetheme{Copenhagen}
%\usetheme{Darmstadt}
%\usetheme{Dresden}
%\usetheme{Frankfurt}
%\usetheme{Goettingen}
%\usetheme{Hannover}
%\usetheme{Ilmenau}
%\usetheme{JuanLesPins}
%\usetheme{Luebeck}
\usetheme{Madrid}
%\usetheme{Malmoe}
%\usetheme{Marburg}
%\usetheme{Montpellier}
%\usetheme{PaloAlto}
%\usetheme{Pittsburgh}
%\usetheme{Rochester}
%\usetheme{Singapore}
%\usetheme{Szeged}
%\usetheme{Warsaw}

% As well as themes, the Beamer class has a number of color themes
% for any slide theme. Uncomment each of these in turn to see how it
% changes the colors of your current slide theme.

%\usecolortheme{albatross}
%\usecolortheme{beaver}
%\usecolortheme{beetle}
%\usecolortheme{crane}
%\usecolortheme{dolphin}
%\usecolortheme{dove}
%\usecolortheme{fly}
%\usecolortheme{lily}
%\usecolortheme{orchid}
%\usecolortheme{rose}
%\usecolortheme{seagull}
%\usecolortheme{seahorse}
%\usecolortheme{whale}
%\usecolortheme{wolverine}

%\setbeamertemplate{footline} % To remove the footer line in all slides uncomment this line
%\setbeamertemplate{footline}[page number] % To replace the footer line in all slides with a simple slide count uncomment this line

%\setbeamertemplate{navigation symbols}{} % To remove the navigation symbols from the bottom of all slides uncomment this line
}

\usepackage[yyyymmdd,hhmmss]{datetime}
\usepackage{hyperref}
\usepackage{cleveref}
\usepackage{autonum}
\usepackage{amsmath}
\usepackage{amsfonts}
\usepackage{amssymb}
\usepackage{mathtools}
\usepackage{parskip}
\usepackage{microtype}
\usepackage[margin=10pt,font=small,labelfont=bf,labelsep=colon]{caption}
\usepackage[margin=10pt,font=small,labelfont=bf,labelsep=space]{subcaption}
\usepackage{color}
\usepackage{wrapfig}
\usepackage{floatrow}
\usepackage{tabularx}
\usepackage{tabulary}
%\usepackage{multicol}
%\usepackage{longtable}
%\usepackage{booktabs} % Allows the use of \toprule, \midrule and \bottomrule in tables
\usepackage{enumerate} 
\usepackage[inline]{enumitem}
\usepackage{url}
\usepackage{marvosym}
\usepackage{wasysym}

\usepackage{graphicx}
\usepackage{epstopdf}

%\usepackage{svg}
%\setsvg{inkscape=inkscape -z -D}

\usepackage{tikz}
\usetikzlibrary{calc}
\usetikzlibrary{arrows}
%\usetikzlibrary{positioning}

\usepackage{algorithm}
%\usepackage{algorithm2e}
\usepackage{algorithmic}
%\usepackage{algorithmicx}

%\usepackage{attachfile}
%\usepackage{minted}
%\usepackage[]{algorithm2e}

\graphicspath{{figures/}}
\floatsetup[table]{capposition=top}
\allowdisplaybreaks
\renewcommand{\dateseparator}{--}

% Input paths
\makeatletter
\def\input@path{{figures/},{sections/}}
\makeatother

\makeatletter
\g@addto@macro{\UrlBreaks}{\UrlOrds}
\makeatother


%-------------------------------------------------------------------------------
% TITLE PAGE
%-------------------------------------------------------------------------------
\title[Short title]{Multi-agent simulation} % The short title appears at the bottom of every slide, the full title is only on the title page
\author{Jaan Tollander de Balsch} % Your name
% Your institution as it will appear on the bottom of every slide, may be shorthand to save space
\institute[Aalto University] 
{Aalto University \\ \medskip\textit{de.tollander@aalto.fi}}
\date{\today} % Date, can be changed to a custom date

\begin{document}

\begin{frame}
\titlepage % Print the title page as the first slide
\end{frame}

\begin{frame}
\frametitle{Overview} % Table of contents slide, comment this block out to remove it
\tableofcontents % Throughout your presentation, if you choose to use \section{} and \subsection{} commands, these will automatically be printed on this slide as an overview of your presentation
\end{frame}

%-------------------------------------------------------------------------------
% PRESENTATION SLIDES
%-------------------------------------------------------------------------------

%------------------------------------------------
\section{Introduction}
%------------------------------------------------
\begin{frame}
\frametitle{Introduction}

A force based simulation model for crowd dynamics. i.e movement of people (agents) is modeled by hypotetical social force. Reminds of particle simulation. 

Game theoretical model is used to model agent behavior in egress congestion (eq. evacuation).

\end{frame}


%------------------------------------------------
\section{Simulation model}
%------------------------------------------------
\subsection{Agent model}
\begin{frame}
\frametitle{Simulation model}
\framesubtitle{Agent model}

\resizebox{\linewidth}{!}{%
\begin{tikzpicture}
% Grid (x, y)
\definecolor{light-gray}{gray}{0.85}
\coordinate[] (p1) at (-8, -5);
\coordinate[] (p2) at ( 8,  5);
\draw[help lines, step=1, light-gray] (p1) grid (p2);

\coordinate[] (xp) at ($ (p1) + (1, 0) + (1, 1) $);
\coordinate[] (yp) at ($ (p1) + (0, 1) + (1, 1) $);
\node[anchor=north] (x) at (xp)  {$ x $};
\node[anchor=east] (y) at (yp) {$ y $};
\draw[<->] (xp) -- ($ (p1) + (1, 1) $) -- (yp);

\section{Agents}
\subsection{Properties}

\begin{table}[H]
\begin{tabularx}{1.0\linewidth}{ l r r r r r}
& Total && Torso & Shoulder & \\
\hline
& $ r $ & $ \pm $ & $ k_{t} = \frac{r_{t}}{r} $ & $ k_{s} = \frac{r_{s}}{r} $ & $ k_{ts} = \frac{r_{ts}}{r} $ \\
\hline\hline
adult & $ 0.255 $ & $ 0.035 $ & $ 0.5882 $ & $ 0.3725 $ & $ 0.6275 $ \\
child & $ 0.210 $ & $ 0.015 $ & $ 0.5714 $ & $ 0.3333 $ & $ 0.6667 $ \\
eldery & $ 0.250 $ & $ 0.020 $ & $ 0.6000 $ & $ 0.3600 $ & $ 0.6400 $ \\
female & $ 0.240 $ & $ 0.020 $ & $ 0.5833 $ & $ 0.3750 $ & $ 0.6250 $ \\
male & $ 0.270 $ & $ 0.020 $ & $ 0.5926 $ & $ 0.3704 $ & $ 0.6296 $ \\
\hline
\end{tabularx} 
\caption{Shoulder, torso and total radii.}
\end{table}


\begin{table}[H]
\begin{tabularx}{1.0\linewidth}{ l l l l }
\hline
\hline
$ r $                    & $ \mathrm{m} $ &  & Total radius \\
$ r_{t} $                & $ \mathrm{m} $ &  & Torso radius \\
$ r_{s} $                & $ \mathrm{m} $ &  & Shoulder radius \\
$ r_{ts} $                & $ \mathrm{m} $ &  & Distance from torso to shoulder \\
$ m $                    & $ \mathrm{kg} $ & $ 80 $ & Mass \\
$ I $                    & $ \mathrm{kg \cdot m^{2}} $ & $ 4.0 $ & Rotational moment \\
\hline
\hline
$ \mathbf{x} $           & $ \mathrm{m} $ &  & Position \\
$ \mathbf{v} $           & $ \mathrm{m} / \mathrm{s} $ &  & Velocity \\
$ v_{0} $                & $ \mathrm{m} / \mathrm{s} $ &  & Goal velocity \\  
$ \hat{\mathbf{e}}_{0} $ &  &  & Goal direction \\
$ \hat{\mathbf{e}} $     &  &  & Target direction \\
\hline
\hline
$ \varphi $              & $ \mathrm{rad} $ & $ [-\pi, \pi] $ & Body angle \\
$ \omega $               & $ \mathrm{rad} / \mathrm{s} $ &  & Angular velocity \\
$ \varphi_{0} $          & $ \mathrm{rad} $ & $ [-\pi, \pi] $ & Target angle \\
$ \omega_{0} $           & $ \mathrm{rad} / \mathrm{s} $ & $ 0.4\pi $ & Max angular velocity \\
%$ \tilde{\omega}_{0} $   &  &  & Target angular velocity \\
\hline
\hline
$ p $                    &  & $ 0 - 1 $ & Herding tendency \\
\hline
\hline
\end{tabularx}
\caption{Properties}
\end{table}


\subsection{Models}
\subsubsection{Circular}

\begin{table}[H]
\begin{tabularx}{1.0\linewidth}{ll}
\hline
\hline
$ \tilde{\mathbf{x}} = \mathbf{x}_{i} - \mathbf{x}_{j} $ & Relative position \\
$ \tilde{\mathbf{v}} = \mathbf{v}_{i} - \mathbf{v}_{j} $ & Relative velocity \\
\hline
\hline
\end{tabularx}
\caption{Relative}
\end{table}

\begin{table}[H]
\begin{tabularx}{1.0\linewidth}{ll}
\hline
\hline
$ d = \left\|\tilde{\mathbf{x}}\right\| $ & Distance\\
$ \hat{\mathbf{n}} = \tilde{\mathbf{x}} / d $ & Normal vector \\
$ \hat{\mathbf{t}} = R(-90^{\circ}) \cdot \hat{\mathbf{n}} $ & Tangent vector \\
\hline
\hline
\end{tabularx}
\end{table}

Total radius and relative distance
\begin{align}
\tilde{r} &= r_{i} + r_{j} \\
h &= d - \tilde{r}
\end{align}


\subsubsection{Three circles}

\begin{align}
\mathbf{x}_{r} &= \mathbf{x}_{c} + \hat{\mathbf{t}} r_{ts}  \\
\mathbf{x}_{l} &= \mathbf{x}_{c} - \hat{\mathbf{t}} r_{ts} \\
\hat{\mathbf{t}} &= \begin{bmatrix} -\sin(\varphi) & \cos(\varphi) \end{bmatrix}
\end{align}

\begin{align}
\mathbf{r}_{tot} = \begin{bmatrix} r_{t} & r_{s} & r_{s} \end{bmatrix}_{i} + \begin{bmatrix} r_{t} \\ r_{s} \\ r_{s} \end{bmatrix}_{j}
\end{align}

\begin{align}
\mathbf{d} &= \left\|\begin{bmatrix} \mathbf{x}_{c} & \mathbf{x}_{r} & \mathbf{x}_{l} \end{bmatrix}_{i} - \begin{bmatrix} \mathbf{x}_{c} \\ \mathbf{x}_{r} \\ \mathbf{x}_{l} \end{bmatrix}_{j}\right\|
\\
&= \left\|\begin{bmatrix} 0 & \hat{\mathbf{t}} r_{ts} & -\hat{\mathbf{t}} r_{ts} \end{bmatrix}_{i} - \begin{bmatrix} 0 \\ \hat{\mathbf{t}} r_{ts} \\ -\hat{\mathbf{t}} r_{ts} \end{bmatrix}_{j}  + (\mathbf{x}_{i} - \mathbf{x}_{j}) \right\|
\\
&= \left\|\begin{bmatrix} 0 & 1 & -1 \end{bmatrix} \left(\hat{\mathbf{t}} r_{ts}\right)_{i} - \begin{bmatrix} 0 \\ 1 \\ -1 \end{bmatrix} \left(\hat{\mathbf{t}} r_{ts}\right)_{j}  + \tilde{\mathbf{x}} \right\|
\\
&= \left\| \mathbf{k} \left(\hat{\mathbf{t}} r_{ts}\right)_{i} - \mathbf{k}^{T} \left(\hat{\mathbf{t}} r_{ts}\right)_{j}  + \tilde{\mathbf{x}} \right\|
\\
&= \left\| \mathbf{c}_{i} - \mathbf{c}_{j}^{T}  + \tilde{\mathbf{x}} \right\|
\end{align}

\begin{align}
\mathbf{h} = \mathbf{d} - \mathbf{r}_{tot}
\end{align}

\begin{enumerate}
\item Find 
\begin{align}
h = \min(\mathbf{h})
\end{align}
and track minimizing values
\begin{align}
\hat{\mathbf{e}}_{ij}, k_{i}, k_{j}, r_{i}, r_{j}
\end{align}

\item 

\begin{align}
\mathbf{r}_{i}^{moment} &= \mathbf{x}_{i}^{c} + k_{i} \cdot \hat{\mathbf{t}}_{i} r_{i}^{ts} + r_{i} \hat{\mathbf{e}}_{ij} \\
\mathbf{r}_{j}^{moment} &= \mathbf{x}_{j}^{c} + k_{j} \cdot \hat{\mathbf{t}}_{j} r_{j}^{ts} - r_{j} \hat{\mathbf{e}}_{ij}
\end{align}

\item Return $ (\tilde{\mathbf{x}}, r_{tot}, h, \mathbf{r}_{i}^{moment}, \mathbf{r}_{j}^{moment}) $

\end{enumerate}



\end{tikzpicture}
}

\end{frame}

\subsection{Social force model}
\begin{frame}
\frametitle{Simulation model}
\framesubtitle{Social force model}

Total force exerted on the agent is the sum of movement adjusting, social and contact forces between other agents and wall.
\begin{align}
\mathbf{f}_{i}(t) = \mathbf{f}_{i}^{adj} + \sum_{j\neq i}^{} \left(\mathbf{f}_{ij}^{soc} + \mathbf{f}_{ij}^{c}\right) + \sum_{w}^{} \left(\mathbf{f}_{iw}^{soc} + \mathbf{f}_{iw}^{c}\right) + \boldsymbol{\xi}_{i}
\end{align}

\pause

Force adjusting agent's movement towards desired in some characteristic time \begin{align}
\mathbf{f}^{adj} &= \frac{m}{\tau^{adj}} (v_{0} \cdot \hat{\mathbf{e}} - \mathbf{v}) 
\end{align}

\end{frame}


\begin{frame}
Velocity dependent algorithm
\begin{align}
\mathbf{f}^{soc} &= -\nabla_{\tilde{\mathbf{x}}} E(\tau)  \\
&= -\nabla_{\tilde{\mathbf{x}}} \left(\frac{k}{\tau^{2}} \exp \left( -\frac{\tau}{\tau_{0}} \right) \right) \\
\\
&= - \left(\frac{k}{a \tau^{2}}\right) 
\left(\frac{2}{\tau} + \frac{1}{\tau_{0}}\right) 
\exp\left (-\frac{\tau}{\tau_{0}}\right )
\left(\tilde{\mathbf{v}} -\frac{a \tilde{\mathbf{x}} + b \tilde{\mathbf{v}}}{d} \right),
\end{align}
where
\begin{align}
a &= \tilde{\mathbf{v}} \cdot \tilde{\mathbf{v}} \\
b &= -\tilde{\mathbf{x}} \cdot \tilde{\mathbf{v}} \\
c &= \tilde{\mathbf{x}} \cdot \tilde{\mathbf{x}} - \tilde{r}^{2} \\
d &= \sqrt{b^{2} - a c}, \quad b^{2} - a c > 0 \\
\tau &= \frac{b - d}{a} > 0.
\end{align}
\end{frame}


\subsection{Rotational motion}
\begin{frame}
\frametitle{Simulation model}
\framesubtitle{Rotational motion}

\end{frame}


\subsection{System of differential equations}
\begin{frame}
\frametitle{Simulation model}
\framesubtitle{System of differential equations}

\end{frame}


\subsection{Game theoretical model}

\begin{frame}
\frametitle{Simulation model}
\framesubtitle{Game theoretical model}

\end{frame}



%------------------------------------------------
\section{Simulations}
%------------------------------------------------
\subsection{Egress congestion}
\begin{frame}
\frametitle{Simulations}
\framesubtitle{Egress congestion}

\end{frame}

%------------------------------------------------


\begin{frame}
\frametitle{Simulations}

\end{frame}

%------------------------------------------------

%\begin{frame}
%\frametitle{Multiple Columns}
%\begin{columns}[c] % The "c" option specifies centered vertical alignment while the "t" option is used for top vertical alignment
%
%\column{.45\textwidth} % Left column and width
%\textbf{Heading}
%\begin{enumerate}
%\item Statement
%\item Explanation
%\item Example
%\end{enumerate}
%
%\column{.5\textwidth} % Right column and width
%Lorem ipsum dolor sit amet, consectetur adipiscing elit. Integer lectus nisl, ultricies in feugiat rutrum, porttitor sit amet augue. Aliquam ut tortor mauris. Sed volutpat ante purus, quis accumsan dolor.
%
%\end{columns}
%\end{frame}



%------------------------------------------------

%\begin{frame}
%\frametitle{Theorem}
%\begin{theorem}[Mass--energy equivalence]
%$E = mc^2$
%\end{theorem}
%\end{frame}

%------------------------------------------------

%\begin{frame}[fragile] % Need to use the fragile option when verbatim is used in the slide
%\frametitle{Verbatim}
%\begin{example}[Theorem Slide Code]
%\begin{verbatim}
%\begin{frame}
%\frametitle{Theorem}
%\begin{theorem}[Mass--energy equivalence]
%$E = mc^2$
%\end{theorem}
%\end{frame}\end{verbatim}
%\end{example}
%\end{frame}

%------------------------------------------------

%\begin{frame}[fragile] % Need to use the fragile option when verbatim is used in the slide
%\frametitle{Citation}
%An example of the \verb|\cite| command to cite within the presentation:\\~
%
%This statement requires citation \cite{p1}.
%\end{frame}

%------------------------------------------------

\begin{frame}
\frametitle{References}
\footnotesize{
\begin{thebibliography}{99} % Beamer does not support BibTeX so references must be inserted manually as below
\bibitem[Smith, 2012]{p1} John Smith (2012)
\newblock Title of the publication
\newblock \emph{Journal Name} 12(3), 45 -- 678.
\end{thebibliography}
}
\end{frame}

%------------------------------------------------

\begin{frame}
\Huge{\centerline{The End}}
\end{frame}

%----------------------------------------------------------------------------------------

\end{document} 